\documentclass[11pt]{amsart}
\usepackage{geometry}
\geometry{letterpaper}
\usepackage{graphicx}
\usepackage{amssymb}
\usepackage{epstopdf}
\usepackage{listings}
\usepackage{color}
\DeclareGraphicsRule{.tif}{png}{.png}{`convert #1 `dirname #1`/`basename #1 .tif`.png}
\title{Similar Days at Airports in the New York Area}
\author{Akhil Shah, Kenneth Kuhn, Chris Skeels}
%\date{}
\begin{document}
\maketitle
\section*{Executive Summary}
\section*{Introduction and Context}
Personnel at the Federal Aviation Administration�s Air Traffic Control System Command Center and at airline operations centers regularly implement Air Traffic Flow Management Initiatives (ATFMIs) purposefully delaying, canceling, and rerouting flights. These initiatives increase the safety and efficiency of the nation�s air transportation system, for example by replacing airborne delay with ground delay, and are necessary during inclement weather and in other situations where demand for system resources exceeds capacity.  In particular, problems at airports often create the need for ATFMIs.  Analysis of the past use of ATFMIs can demonstrate the relative success of courses of action but must account for the distinct conditions faced during planning and operations.  An identification of days that are similar can help, for example allowing analysts to focus on the 10 days in the past two years when there was thunderstorm activity at the key airports in the New York area between 8am and 11am, local time, but clear weather the rest of the day.

This report describes our work to develop methodologies for the identification similar days in terms of aviation weather and air traffic operations at the airports in the New York area.  There are many reasons why an analyst may want to identify similar days and thus there are multiple ways to arrive at a definition of similar days.  This report follows an earlier report to identify similar days based on conditions in the airspace around New York City.  We do not wish to replicate the prior report and thus only report on new findings specific to our study of airports.  The earlier report includes more detail regarding why it would be beneficial to identify similar days from the perspective of ATFMI planning or operations.

As in our earlier work that focused on the airspace, we have published many of our results focusing on airports in a web based application.  The application is essentially identical to the one we developed and reported on previously.  Please see the earlier report for a description of our web based application.

In this report, we report on our work to identify features that describe aviation weather and air traffic at airports in the New York area, the data sets we use to define these features, and interesting results we obtain.  The final section of this report is a conclusion that details how the airport-focused results presented here can be blended with airspace-focused results obtained previously.

\section*{Selecting Features to Describe Airport Conditions}

\subsection*{Knowledge-Based Feature Selection}
There are lots of features here.

\section*{Airport Weather and Air Traffic Data}
We spit the types of data of interest into data that describe conditions at airports versus in the airspace, as well as those that describe weather versus those that describe scheduled operations.

Airports themselves issue Terminal Area/Aerodrome Forecast reports (TAFs) and Aviation Routine Weather Reports (METARs) which summarize local forecast and observed weather conditions. METARs can contain select forecast data but, generally speaking, TAF data are forecast data while METAR data are observational data. TAF and METAR data contain information on: wind speed and direction, wind gusts, visibility, precipitation, cloud height, cloud cover, humidity, and pressure. Prior research efforts have linked many of these variables to traffic flow management initiatives. Smith and Sherry (2008) successfully use TAF data to forecast Ground Delay Program initiation, without giving details on the relative importance of specific variables. Mukherjee et al. (2014) point out the relevance of hourly observations of visibility, cloud height, wind, convection, and precipitation in particular, again for predicting GDPs. TAFs and METARs are issued roughly hourly to ensure reports keep up with changing weather conditions but also that distinct consumers of the data have a consistent report and time to plan against it.

The studies cited above, along with Liu et al. (2014) focus on forecasts, typically forecast conditions two hours ahead of time. Different TFMIs are implemented at different time scales, e.g., flow metering is typically used to respond to short-term supply-capacity imbalances while Ground Delay Programs are used to mitigate the potential for a larger problem further in the future. It is not obvious that two-hour forecast data is the ideal for our efforts. There is actually a larger question as to the relative importance of differences between weather forecast and observational data here. When determining that January 1, 2012 is similar to January 2, 2012 but not to July 1, 2012, it may or may not matter a great deal whether forecast or observed weather data are used. DeLaura and Allan (2003) show how, for analyses they perform, differences between forecast and observed weather patterns are relatively unimportant. Their conclusion is that: �for air traffic flow management tools, a realistic operational model may be at least as important as the frequently discussed problem of weather forecast uncertainty.� This is a somewhat surprising statement given that the authors are concerned with route guidance, where small details can be important. We do note that the authors do not claim that forecast uncertainty is completely unimportant.

For now, we note that a collection of hourly observations of TAF and METAR data, or other data detailing the variables included in TAF and METAR data, covering the busiest airports in the New York area over an extended period of time would comprise an ideal data set to describe airport weather in the area at the time.

Traffic flow management is concerned with mitigating temporary supply � demand imbalances. At the level of an airport, the primary concern is almost always that the number of aircraft scheduled to land at and take off across a set block of time may be higher than the throughput of the runways will allow. A reasonable length for such a block of time would be an hour or a few hours. Supply � demand imbalances over shorter periods of time, e.g., two aircraft scheduled to land at the same time, can be accommodated with minor path adjustments rather than the more strategic traffic flow management initiatives. Hourly observations of scheduled operations could be easily compared to weather data such as TAFs and METARs that are reported hourly. A collection of hourly observations of the numbers of aircraft scheduled to land at and take off from the busiest airports in the New York area would be an ideal data set here.

Our definition of the ideal data set for describing airport weather referenced TAFs and METARs, two formats airports currently use in their reports. Current TAF and current METAR data are available. METAR data has been collected in a publicly accessible historical archive. Unfortunately, we have not identified a similar large archive of historical TAF data. This result is unsurprising; there are relatively many uses for historical observations of weather conditions and relatively few for historical forecasts. Luckily, there is an alternate data source that does contain both forecast and observed weather at airports. The National Oceanic and Atmospheric Administration�s (NOAA�s) Localized Aviation MOS (Model Output Statistics) Product (LAMP) combines much of the same data that appears in TAF and METAR data. This data is free to download from NOAA. Table 1 summarizes the availability of useful airport weather data.
\section*{Results and Discussion}
Note that

\section*{Conclusion}
In conclusion,

\bibliographystyle{abbrv}
\bibliography{all}
\end{document} 